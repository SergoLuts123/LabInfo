\documentclass{beamer}
\usepackage[english,russian]{babel}
\usepackage[utf8]{inputenc}

% Стиль презентации
\usetheme{Copenhagen}
\begin{document}
\title{Билет №10} 
\author{Луценко Сергей Викторович}
\institute{СПбГЭТУ «ЛЭТИ»}
\date{Санкт-Петербург, 2019} 
% Создание заглавной страницы
\frame{\titlepage} 
% Автоматическая генерация содержания
\begin{frame}{Содержание}
\begin{thebibliography}{10}
\beamertemplatebookbibitems

\bibitem{1}
{\sc 1}. {\em История возникновения Редактора VI.}
\bibitem{2}
{\sc 2}. {\em Интерфейс.}
\bibitem{3}
{\sc 3}. {\em Режимы Редактора VI.}
\bibitem{4}
{\sc 4}. {\em Возможности Vi:}
\bibitem{5}
{\sc 5}. {\em Редактор VI-как текстовый процессор. Работа с Редактором VI.}
\bibitem{6}
{\sc 6}. {\em Заключение.}
\end{thebibliography}
\end{frame}

\begin{frame}{История возникновения Редактора VI}
В то время наиболее распространённым был редактор ed. Поскольку он был довольно сложным для «простого смертного», George Coulouris разработал редактор em (editor for mortals — редактор для смертных). Билл Джой модифицировал редактор em и назвал его en, а позднее — он получил название ex, на котором и основан vi.
\end{frame}

\begin{frame}{Интерфейс}
В отличие от многих привычных редакторов, vi имеет модальный интерфейс. Это означает, что одни и те же клавиши в разных режимах работы выполняют разные действия.Vi называется экранным редактором, поскольку использует в качестве рабочего поля весь экран терминала, а не одну строку, как ed. Экран задействован большей частью для отображения редактируемого текста, а одну строку – последнюю – vi отводит для общения с пользователем в стиле ed. Помимо понятия текущей строки в vi появляется текущая позиция в строке, указателем на которую работает курсор. Все команды vi применяются к тексту именно в том месте, где находится курсор.


\end{frame}

\begin{frame}{Режимы Редактора VI.}
Редактор VI имеет два основных режима: 
\newline



1.  Командный  -  в  этом  режиме можно перемещаться по файлу и
выполнять редактирующие команды над текстом. Команды вызываются
обычными латинскими буквами.

2. Режим ввода текста - в этом режиме обычные латинские  буквы  будут вставляться в текст.



\end{frame}

\begin{frame}{Возможности Vi:}
- Работа с несколькими текстами одновременно в разных окнах.
\\
- Справка по работе с редактором с описанием синтаксиса.
\\
- Подсветка синтаксиса.
\\
- Поддержка языка сценариев.
\\
- Создание модулей расширения — плагинов.
\\
- Автоматическое продолжение команд, слов и имён файлов.
\\
- Автоматический вызов внешних команд.
\\
- Распознавание и преобразование файлов различных форматов.
\\
- Запись и исполнение макросов.
\\
- Поддержка языков с письмом справа налево (арабских и других).
\\
- Сворачивание (folding) текста для лучшего обзора.
\\
- Графический интерфейс в специальных версиях(GTK,Motif)
\\
А также другие.

\end{frame}
\begin{frame}{Редактор VI-как текстовый процессор. Работа с Редактором VI.}

Сам vi (от visual editor) является визуальным режимом строкового редактора ex ("экс"). Когда ещё не было (больше 40 лет назад) дисплеев, компьютеры выводили строки на печатающие терминалы. Оператор давал редактору команды правки строк, компьютер печатал на бумаге только исправленные строки. "Визуальность" редактора vi здесь означает возможность показа экранной страницы и мгновенного (насколько возможно) отображения правок, то есть более наглядный способ работы.
\end{frame}

\begin{frame}{Редактор VI-как текстовый процессор. Работа с Редактором VI.}

Часто используемые команды:

-/str — Поиск строки str вперед. str может быть регулярным выражением
\\
-?str — Поиск строки str назад
\\
-n — Повторить поиск в том же направлении
\\
-N — Повторить поиск в обратном направлении
\\
-:e! — перезагрузить текущий файл
\\
-:33 — переместиться на 33-ю строку текстового файла
\\
-i — перейти в режим редактирования
\\

\end{frame}
\begin{frame}
-a — перейти в режим редактирования после текущего символа
\\
-u — отменить последнее действие
\\
-. — повторить последнее действие
\\
-x — вырезать символ под курсором
\\
-yy — копировать строку
\\
-dd — вырезать строку
\\
-p — вставить
\\
-J — склеить две строки
\\
-:w — сохранить файл на диске
\\
-:wq — выход с сохранением файла (shift + zz)
\\
-:q — выход
\\
-:q! — выход без сохранения файла (shift + zq)
\\
-:r — вставить в документ другой файл



\end{frame}

\begin{frame}{{Редактор VI-как текстовый процессор. Работа с Редактором VI.}}
Так же существуют клоны VI:
\newline


-Vim
\\
-elvis
\\
-vile (англ.)
\\
-nvi (англ.) — реимплементация vi под свободной лицензией BSD
\\
-viper — плагин для Emacs, эмулирующий команды vi

\end{frame}

\begin{frame}{Заключение}
Подводя итог, можно выявить 3 основных преимущества текстового редактора VI:
\newline


\begin{enumerate}
\item Широкий спектр возможностей.
\item Модальный интерфейс.
\item Простой для запоминания и понимания список команд.
\end{enumerate}
\end{frame}



\end{document}